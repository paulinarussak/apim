
\section{APIM Tools}

\begin{itemize}
\item APIM checklist

	In both processes described as examples we performed APIM. To keep track of actions performed we could make use of APIM checklist to have an overview on each APIM step and not forget about anything.

\item Audit Schedule, Plan, Checklist, Report

	In Post Maturity - Improve phase of APIM we could perform audit and develop continuous process improvement plan. Tools like audit schedule, plan, checklist and report will enable us to do that correctly and easier our work. Performing audit will help us ensure that created processes are being used. Creating audit schedule allows us to gain overview of distribution of actions in time. In our example (process 1) audit plan would be created and used by internal team of expert. They should take into consideration if completion criteria were met - in this case if the overtime working hours were reduced and if plans/procedures defined during prematurity phase were realized. One important aspect is to verify if all of the defined "involved people" were actually involved - Product Owner and developers team. Audit report will let us show results of audit in clear and concise way. It should contain comments and conclusions coming from the audit. Audit checklist (or Gap Analysis) would be other useful tools to look at while creating the audit report. In our example audit report could contain information about overtime hours reduction or if Product Owners are actually using newly developed tools - sometime even after process improvement actions people tend to go back to old habits. Following standardized templates of these documents will allow to easily compare audit results with previous ones or possible next ones.

\item Initial Executive Kickoff Meeting Agenda and Minutes

	These two tools are great to attract and keep executive staff attention and focus. Presenting agenda before the meeting let people realize purpose of this meeting, they can be prepared for what's it going to be. Example agenda (for our process 3) could contain bullets like: "vision: eliminate issues with emergency situations" or "objective: reduce to zero possibility of accident happening when no support is available". Meeting Minutes should be created to ensure that everyone that participated in the meeting is on the same page, understand everything the same way, to prevent misunderstanding. Minutes from this meeting should contain different items like attendees, discussions or action items. Example action items for process 3: "measure current time when projects lack support" or "spread knowledge about projects across team". Both of these documents should be kept as short and as simple as possible - the shortest the documents the more probable someone actually reads them.

\item Process Group Meeting 
\end{itemize}
