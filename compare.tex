\section{Compare GQM+ with APIM}

In this section we want to focus on comparing APIM and GQM approaches. We describe process improvement methods for Project 1 using GQM and APIM. Both methodologies are based on continuous improvement approach. In GQM we work in cycles based on develop-implement-learn cycle. APIM from another hand is more iterative. In each iteration we need to choose the new mini goal that brings us closer to reached ultimate goal.

GQM+ Strategies are very focused on metrics. This methodology provide the great tool and mindset to create right metrics that provide us valuable information. There is also some risk in that situation - too many metrics can increase complexity of improvement process. 

APIM is mote about making improvements in small steps. We need to "break the elephant" and implement the improvements in compact, valuable parts. APIM needs metrics but do not provide tools to help creating the metrics. Based on this conclusion both methods should be used together. GQM provide more defined improvement methods. APIM is also about mind settings.

In the sections above we presented the APIM and GQM methods for project 1 separately. Now we want to propose how we can use GQM and APIM together. We really like the idea of working in iteration (what was presented in APIM), so we want to use the overall model from APIM. In the Pre-maturity phase we want to expand planning part of GQM+ Strategies. Those strategies provide better action list and we gain in the same time also the valuable metrics and overall view of the current situation. Next in Maturity phase, when we need to create action plan to the each actions from action list. We can use it GQM grid that helps us to create right strategies and again provide valuable metrics. 

Presented above approach is one of possible combination of APIM and GQM+ Strategies. What is more important those methods are complementary and should be use together, because both provide the hight quality solutions.