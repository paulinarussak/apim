\section{Processes description}

In this section we want to present the processes that we find in our project and what improvements should be done. Analyze of improvements contains improvements descriptions and predicted obstacles.

\subsection*{Process 1}

First process comes from project that one of us was participate in a big telecommunication company as a software developer. There was a few processes in that project, but we want to focus on process that handle the tasks flow in the system. This project was running in some kind of Scrum. At the beginning of the release Product Owner (PO) gets a list of requirements that should be done till specified date by ten teams. Each team has 6-8 members. Product Owner was setting each requirement in preliminary order and attributed tasks to teams. Each team take care of one domain. After this initial settings Product Owner gives a list of requires that can be handle in this release to Product Manger and release starts.

Project is running in two-weeks sprints. At the beginning of each sprint Product Owner gives a list of requirements or User Stories to the team. Team make a Scrum Poker and based on velocity and intuition select the proper amount of tasks to the sprint. When Product Owner accept the Sprint Backlog, sprint starts. As was mentioned above sprint takes two weeks and included activities such as: develop, test, integration. 

After sprint finish, team meets the Product Owner and shows him the work that is done and undone, velocity and time that team spends on maintenance. After this review section teams goes to the Retrospective meeting and Product Owner update his scheduled list and prepare new requirements to the next sprint.

Let's focus on Product Owner work and process from his perspective in that case. List of requirements is delivered in excel files and this is the main tool to schedule tasks. Each requirement has estimated time execution time (in hours) and the domain that should be done in. Teams during planning gives the PO their estimates (in story points) and then after each sprint they gives the PO number of effort that left.  Some of requirements should be done in particular order, and some of tasks appears in project late (because depends on external work). So there is a lot of issues that PO should concern about. In project there are two Product Owners, so they have to schedule work for five teams each in excel.

Based on assumption above there is a problem in planning part. First of all PO have to spend a lot of time to schedule tasks properly in excel. He don't see the whole picture, because of size of the project, so there are situations that team have anything to do in two sprints, because some other tasks are late and then they have to work overtime. 

Our proposition to improve is to give the POs new tools to schedule tasks. This tool should help with daily work and take care about iteration work. There is one big obstacle: people have a tendency to do exactly what is on the tools output. To avoid that PO should gain at least two different way to schedule work. It force him to think about the proposed solution, alternatively add his own improvements and choose the best option. 

\subsection*{Process 2}

Both of us was participating in a university project about Scrum. Project was about creating a web solution system to handling bank operations. We work in group of six and that contain people from different countries and cultures. 

The project flow starts from overall planning, during which the team describes the main project goal and set initial Project Backlog. After discussion with customer about vision of the product and priority, project starts. First iteration begins from Scrum Poker, where team estimate effort of particular User Stories, after this team decides which US will be in commitment and discusses about Sprint Backlog this with customer.

The next step is a developing part. During this part team starts coding tasks based on DoD. Members pick tasks from Sprint Board from column "TO DO" and then move them to the text columns: "DOING", "IN REVIEW", "DONE". Because of accommodation issues team members can't meet face to face, so there was remote work. Four times in sprint there was daily session by chat. During this session all  coworkers answer three question:
\begin{inparaenum}
\item "what did I do?",
\item "what will I do?",
\item "what obstacles do I have?"
\end{inparaenum} and discuss of necessary. \textit{może coś o pull requestach i code review? NOe wiem jak to ubrać w słowa, żeby nie było masło maslane}

After two weeks development part stops and starts revision and demo part. At the demo team shows work that was done to the customer, has discussion about customer needs and satisfaction, update Product Backlog and vision of the product. Then team goes to the Retrospective meeting and has discussion about good things and bad things that happened in the last sprint and what improvement should be done. In that moment first iteration stops. And second starts from planning again.

In this project there was three iterations. We don't finish all items from Product Backolg, but few more was added. We measure VSM, and it came out that there was some bottlenecks in the system. 

\textit{opis projektu, ustawienia początkowe, jak leciał scrum, zakończenie, probkemy: za mało usawień poczatkowych, brak analizy ryzyka, budzetu, etc,; w trakcie: za mało testów, problemy komunikacyjne}

\subsection*{Process 3}

Third process is related to the experience working as DevOps engineer. The process that we wanted to discuss here is how the team performed at maintenance. Because of the characteristic of DevOps work in this company, this process is more team-oriented rather than project-oriented. Team in this example was very specific - it consisted of one experienced worker (full-time) and six part-time workers (20-30 hours/week). Members of the team were responsible for contributing to several different projects (developing and maintaining them). Each member of the team had different responsibilities, which resulted in highly-specialized workers. This was approach in the company - usually deep knowledge of one or two team members was required, and general idea of the project was enough for the rest of members. It worked quite well in the other teams since they were more mature and all of the members were working full-time. In this specific team that caused a lot of issues to the process of maintenance. Because most of the workers in the team were part-time workers, several times it happened that some of the services provided by the team failed (e.g. because of failure of some external service) and required team member action. Team members that possessed deep knowledge were not available at that time and other team member had to act - this resulted in much longer fixing time (than it would take for experienced member) and of course disturbing the team member's planned work. Approach which seemed to work in different teams did not work in this specific one, the high risk of team member missing was not identified and addressed, knowledge was not spread enough across the team.
