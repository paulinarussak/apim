\section{Processes description}

In this section we want to present the processes that we find in our projects and improvements that we think should be done. Analysis of the improvements contains improvements descriptions and predicted obstacles.

\subsection*{Process 1}

First process is from project that one of us was a part of in a big telecommunication company as a software developer. There were a few processes in that project, but we want to focus on process that handled the tasks flow in the agile project. This project was running in some kind of Scrum. At the beginning of the release Product Owner (PO) gets a list of requirements that should be done till specified date by ten teams. Each team has 6-8 members. Product Owner was setting each requirement in preliminary order and attributed tasks to teams. Each team takes care of one domain. After this initial setting, Product Owner gives a list of requirements that can be handled in this release to Product Manger and the release starts.

Project is running in two-weeks sprints. At the beginning of each sprint Product Owner gives a list of requirements or User Stories to the team. Team plays a Scrum Poker and based on velocity and intuition selects proper amount of tasks for the sprint. When Product Owner accepts a Sprint Backlog, sprint starts. As mentioned above, sprint takes two weeks and includes activities like: developing, testing, integration. 

After sprint finishes, team meets the Product Owner and shows him the work that is done and undone, velocity and time that team spends on maintenance. After this review section team proceeds with Retrospective meeting and Product Owner updates his scheduled list and prepares new requirements for the next sprint.

Let's focus on Product Owner work and process from his perspective. List of requirements is delivered in excel files and this is the main tool used for tasks scheduling. Each requirement has estimated execution time (in hours) and the domain that should be done in. Teams during planning give the PO their estimates (in story points) and then, after each sprint, they give the PO number of effort that is left. Some of the requirements should be done in a particular order, some of the tasks also appear in project late (because they depend on external work). There are a lot of issues that PO should be concerned about. In project there are two Product Owners, so they have to schedule work for five teams each in excel.

Based on assumption above there is a problem in planning part. First of all PO have to spend a lot of time to schedule tasks properly in excel. They don't see the whole picture, because of the size of the project. There are situations that team do not have anything to do in few sprints, because some other tasks are late and then they have too much work and need to work overtime. 

Our proposition to improve is to give the POs new tool to schedule tasks. This tool should help with daily work and take care about iteration work. There is one big obstacle: people have a tendency to do exactly what is shown on the tools output. To avoid that PO should gain at least two different ways to schedule work. It will force him to think about the proposed solutions, alternatively add his own improvements and choose the best option. 

\subsection*{Process 2}

Both of us participated in a university project about Scrum. Project was about creating a web solution system for handling banking operations. We worked in a group of six and it contained people from different countries and cultures. 

The project flow starts from overall planning, during which the team describes the main project goal and set initial Project Backlog. After discussion with customer about vision of the product and priority, project starts. First iteration begins with Scrum Poker, where team estimates effort of particular User Stories, after this team decides which US will be in commitment and discusses about Sprint Backlog this with customer.

The next step is a developing part. During this part team starts coding tasks based on DoD. Members pick tasks from Sprint Board from column "TO DO" and then move them to the text columns: "DOING", "IN REVIEW", "DONE". Because of accommodation issues team members can't meet face to face, so they work remotely. Four times in sprint there is online 'stand-up meeting'. During this session all coworkers answer three question:
\begin{inparaenum}
\item "what did I do?",
\item "what will I do?",
\item "what obstacles do I have?"
\end{inparaenum} and discuss of necessary.
During the development part we use the system version control and to provide the highest quality code we do code reviews and pull requests. 

After two weeks development part stops and starts revision and demo part. At the demo team shows done user stories to the customer, has discussion about customer needs and satisfaction, update Product Backlog and vision of the product. Then team goes to the Retrospective meeting and has discussion about good things and bad things that happened in the last sprint and what improvement should be done. In that moment first iteration stops. And second starts from planning again.

In this project there were three iterations. We didn't finish all of the items from the Product Backolog, but some of them more was added during project lifetime (so final product backlog differed from the initial one). We measured VSM, and it came out that there were some bottlenecks in the system. One of the bottleneck was code review part and integration. This was because only one person handled all those duties. We also found out that we skipped a lot of initial settings. In our project we missed the risk analysis and budget. We also had problems with whitebox tests and some communication issues occured.

\subsection*{Process 3}

Third process is related to the experience working as DevOps engineer. The process that we wanted to discuss here is how the team performed at maintenance. Because of the characteristic of DevOps work in this company, this process is more team-oriented rather than project-oriented. Team in this example was very specific - it consisted of one experienced worker (full-time) and six part-time workers (20-30 hours/week). Members of the team were responsible for contributing to several different projects (developing and maintaining them). Each member of the team had different responsibilities, which resulted in highly-specialized workers. This was common approach in the company - usually deep knowledge of one or two team members was required, and general idea of the project was enough for the rest of members. It worked quite well in the other teams since they were more mature and all of the members were working full-time. In this specific team that caused a lot of issues to the process of maintenance. Because most of the workers in the team were part-time workers, several times it happened that some of the services provided by the team failed (e.g. because of failure of some external dependencies) and required team member action. Team members that possessed deep knowledge were not available at that time and other team member had to act - this resulted in much longer fixing time (than it would take for experienced member) and of course disturbing the team member's planned work. Approach which seemed to work in different teams did not work in this specific one, the high risk of team member missing was not identified and addressed, knowledge was not spread enough across the team.
