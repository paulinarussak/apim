\section{GQM+}

GQM+ for Project 1:

GQM Strategies Element:

\begin{itemize}
\item Organization Goal:
  \begin{itemize}
  \item[\textit{object:}] Planning in software development process \\
  We assumed that planning in our software development process need to be improved.
  \item[\textit{focus:}] effectiveness \\ We want to increase effectiveness of planning. 
  \item[\textit{magnitude:}] Reduce overtimes by X\% \\
  We reach the goal if overtimes will be reduced by X\%.
  \item[\textit{time frame:}] End of the release.\\
  We want to achieve the goal before end of current release.
  \item[\textit{organizational scope:}] Product Owner and development teams \\
  Main success of the achieving the goal depends on Product Owner and development teams.
  \item[\textit{constrains:}] Maintenance tasks \\
  Maintenance tasks occur in the system randomly and may influence to feature work.
  \item[\textit{relationships:}] Decrease defect level, cost reduction. \\
  We want to increase effectiveness of planning part, so we want to have at least stable level of maintenance work, so reaching decrease defect level help us. From another hand we want to achieve the goal by implement new tool, which can affect on budget and be in opposition to cost reduction goal.
  \end{itemize}
\item Context/Assumption: \\
In our organization we use the waterfall in a up level of planning and some variation about Scrum in development part. In this paper we want to focus on development part. List of requirements is known at the beginning of the project, but some requirements occur in the project at a specified time or have to be finish earlier, because of dependencies to other projects. Product Owner has to   create initial scheduling list of requirements and after each sprint he needs to update previous settings. Now he do all that work in excel, so he don't see the whole picture and because of that teams are uneven work loaded. 
\item Strategy (\textit{how?}):\\
The proposed solution is to implement new tool to the planning process. This tool should have to possibility to change scheduling algorithms parameters to customize output. Moreover tool has to provide minimum two version of output to avoid situation where Product Owner unthinkingly use the tool.
\end{itemize}

GQM Graph:
\begin{itemize}
\item[MG:] Equalize work load
	\begin{itemize}
	\item[Q:] How much people work overtime?
    	\begin{itemize}
    	\item[M:] counting overtime hours
    	\end{itemize}
    \item[Q:] How much time team spent on features?
    	\begin{itemize}
    	\item[M:] hours spent on features per sprint,
        \item[M:] overall feature time in project,
    	\end{itemize}
    \item[Q:] How team members feel about work load?
    	\begin{itemize}
    	\item[M:] satisfaction of the team
    	\end{itemize}
	\end{itemize}
\item[MG:] Increase effectiveness of Product Owner work.
	\begin{itemize}
	\item[Q:] How much time Product Owner spend on scheduling tasks?
    	\begin{itemize}
    	\item[M:] time spend on scheduling task at the beginning of the project,
        \item[M:] time spend on scheduling task at the beginning of each sprint,
    	\end{itemize}
    \item[Q:] How proper is plan created by Product Owner?
    	\begin{itemize}
    	\item[M:] number of late tasks,
        \item[M:] overtime hours,
        \item[M:] VSM: waiting time,
    	\end{itemize}
    \item[Q:] How people enjoyed their work?
    	\begin{itemize}
    	\item[M:] satisfaction of team,
        \item[M:] satisfaction of Product Owner,
        \item[M:] satisfaction of customer.
    	\end{itemize}
	\end{itemize}
\end{itemize}