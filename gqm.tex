\section{GQM+}

GQM+ Strategies approach was presented in  book \textit{Aligning Organizations through Measurement : the GQM+ Strategies Approach}. This technique is about reaching the organizational goals through measurements. The Goal in this case is \textit{what} the organization wants to achieve. Strategy is about \textit{how} we want to achieve our goals. Strategies are specific actions. Measurements are collected constantly to support making-decision process and to follow in the right direction \cite{basili}.

The GQM+ Strategies  Process consist of initial phase and six phases that are in cycle organized in three main sections. Let's begin from initializing phase. During initialization phase we need to specify the organization scope that will be improved. Then we need to prepare overall plan to our process, which consist analyze budget, risk, timeline with milestones. At the end of this we have to train involved people and take care about right minds settings.

Process cycle starts from develop section. During this section we create the GQM grid that contains goals, strategies and data to measure. To do this we need to 1) Characterize Environment by describe the current situation in organization. Next 2) Define Goals, Strategies and Measurements by creating organization model grid that contains goals and actions for organization. After development part is ready we goes to the implementation part, where specify the plans in 3) Plan Grid Implementation phase. Phase 4) Execute Plans is about executing and verifying plans. The responsible person can change plans here if it's needed. In the third part we learn by analyze and drawing conclusions. First 5) Analyze Outcomes and by this we mean finding what goes well or wrong and investigate those by root causes analyze. The last thing is collect all data from previous  analyzes and create 6) Package Improvements, that will be use in the next cycle. \cite{basili}

We decide to apply GQM+ Strategy to the Project 1. We create GQM+ Strategies Elements and those are presented below:

\begin{itemize}
\item Organization Goal:
  \begin{itemize}
  \item[\textit{object: (analyze some)}] Planning in software development process \\
  We assumed that planning in our software development process need to be improved.
  \item[\textit{focus: (with respect to)}] effectiveness \\ We want to increase effectiveness of planning. 
  \item[\textit{magnitude:}] Reduce overtimes by X\% \\
  We reach the goal if overtimes will be reduced by X\%.
  \item[\textit{time frame:}] End of the release.\\
  We want to achieve the goal before end of current release.
  \item[\textit{organizational scope: (from the point of view of)}] Product Owner and development teams \\
  Main success of the achieving the goal depends on Product Owner and development teams.
  \item[\textit{constrains:}] Maintenance tasks \\
  Maintenance tasks occur in the system randomly and may influence to feature work.
  \item[\textit{relationships:}] Decrease defect level, cost reduction. \\
  We want to increase effectiveness of planning part, so we want to have at least stable level of maintenance work, so reaching decrease defect level help us. From another hand we want to achieve the goal by implement new tool, which can affect on budget and be in opposition to cost reduction goal.
  \end{itemize}
\item Context/Assumption: \\
In our organization we use the waterfall in a up level of planning and some variation about Scrum in development part. In this paper we want to focus on development part. List of requirements is known at the beginning of the project, but some requirements occur in the project at a specified time or have to be finish earlier, because of dependencies to other projects. Product Owner has to   create initial scheduling list of requirements and after each sprint he needs to update previous settings. Now he do all that work in excel, so he don't see the whole picture and because of that teams are uneven work loaded. 
\item Strategy (\textit{how?}):\\
The proposed solution is to implement new tool to the planning process. This tool should have to possibility to change scheduling algorithms parameters to customize output. Moreover tool has to provide minimum two version of output to avoid situation where Product Owner unthinkingly use the tool.
\end{itemize}

When organization goals and strategies are defined, we need to prepare GQM grid for more specified, measurement goals. Those are presented below:
\begin{itemize}
\item[MG:] Equalize work load
	\begin{itemize}
	\item[Q:] How much people work overtime?
    	\begin{itemize}
    	\item[M:] counting overtime hours
    	\end{itemize}
    \item[Q:] How much time team spent on features?
    	\begin{itemize}
    	\item[M:] hours spent on features per sprint,
        \item[M:] overall feature time in project,
    	\end{itemize}
    \item[Q:] How team members feel about work load?
    	\begin{itemize}
    	\item[M:] satisfaction of the team
    	\end{itemize}
	\end{itemize}
\item[MG:] Increase effectiveness of Product Owner work.
	\begin{itemize}
	\item[Q:] How much time Product Owner spend on scheduling tasks?
    	\begin{itemize}
    	\item[M:] time spend on scheduling task at the beginning of the project,
        \item[M:] time spend on scheduling task at the beginning of each sprint,
    	\end{itemize}
    \item[Q:] How proper is plan created by Product Owner?
    	\begin{itemize}
    	\item[M:] number of late tasks,
        \item[M:] overtime hours,
        \item[M:] VSM: waiting time,
    	\end{itemize}
    \item[Q:] How people enjoyed their work?
    	\begin{itemize}
    	\item[M:] satisfaction of team,
        \item[M:] satisfaction of Product Owner,
        \item[M:] satisfaction of customer.
    	\end{itemize}
	\end{itemize}
\end{itemize}

As we can see above GQM grid provide analyze based on measurement goal (MG). Questions help to better understand the case and provide better measurements. GQM+ Strategies are about  overall organization goals and increase effectiveness for whole organization. GQM grid from the other hand provide monitoring of process, how strategies really works and if goals are reached.