\subsection*{Project 3}

Maintenance is the process we look at in the third project. Applying APIM, we need to start with Launch phase again. To convince the management that the process needs an improvement, we need to present them some data. We can collect measurement, based on the working hours of team members, how much time certain projects are without responsible team members - when some failure happens there is no person responsible for the project that can take care of the accident. To give an example: Jim and Mary are the only workers responsible for project A. During 40 hours workweek either both or one of them is working all the time, excluding time window on Monday where from 12 a.m. to 4 p.m., when none of them is working. This leaves us with 4 hours time window every week, so that if an accident happens, there is nobody responsible to take care of it. We can think of it as 10\% sign chance of this situation happening (this is of course a simplification which takes into account this 4 hours time window and 40 hours workween as the only factors). To further convice the management we can justify that having some services down for longer time can cause financial loss. We can give them real examples of situations that actually happened - like the need of calling the team members that were away on holidays to ask for help.

Brief action plan proposed for this process: \\
\textbf{Title:} Reducing total time of situations where services are live and no responsible team member is currently working.\\
\textbf{Problem definition:} Members of the team are working part-time and sometimes situations happen, that there is nobody to act in case of emergency situation.\\
\textbf{Goal:}  We want to reduce total time of situations like this happening.\\
\textbf{Team:} Project improvement team is needed. It should contain 2 people.\\
\textbf{People involved:} Development team.\\
\textbf{Strategy:} Knowledge about the projects should be spread across the team so that more members are aware of what to do in emergency situations.\\
\textbf{Desired Results:} Total time of service running without emergency help reduced to zero.\\
\textbf{Risk:} We can identify a risk that activities related to spreading knowledge about the projects across the team can take too much time (especially considering part-time workers) which can reduce overall productivity. Another risk is that hypothetically speaking there is a possibility that there still is no member responsible for the service (bad luck in schedule of workers so that nobody from the team is working at certain time, team members illness etc.).\\
%\textbf{Timeline and milestones:} \textit{tu nie rozumiem dokładnie o jaki timeline chodzi}\\
\textbf{Completion Criteria:} Total time of services without emergency help decreased to zero based on team members working hours.
 %To sum up our improvement process is about making improvement in planning part in development cycle. We want to monitor and schedule tasks flow in agile project to equalize the team load. The problem occur because of Product Owners schedule tasks for whole release using excel and because of that they miss the whole picture. Work for teams is unevenly distributed over project life cycle. The new monitoring and schedule method is needed.

Now comes Maturity phase. As described previously, we are following Maturity phase steps:
\begin{enumerate}
\item[Awareness] There is some hindrance in the maintenance process that sometimes causes too long downtimes of services in case of emergency situations. 
\item[Triage] We need to prioritize that will let us measure our current situation.
\item[Resolution] Preparation of action plan for measuring current situation. We need to measure occurences of the aforementioned situations based on team members working hours. This is simple task that can be automated, research needs to be done if proper software exists or if we need to implement it (which shouldn't be a tough task). 
\item[Training] Gathering information about working hours of team members is already in place, so team members do not need additional training. Process improvement team members need to understand the output of measurement software.
\item[Deployment] Deploy information gathering to our projects team.
\item[Trial] Insert collecting data from measurement system into process flow.
\end{enumerate}

After first iteration which was small and simple we have some meaningful data about current state of our processes, we can continue with analyzing it and persist in trying to reach the goal. Possible second iteration:
\begin{enumerate}
\item[Awareness] Analyzing collected data we can find out that each week there are in total 8 hours of situations that we want to avoid. We can notice that it is relevant only for projects X and Y.
\item[Triage] We prioritize knowledge sharing tasks for projects X and Y over other projects. Both projects are similarily important, but for project X total time is 6 hours so we set it to the highest priority.
\item[Resolution] We are preparing action plan for knowledge sharing for projects X and Y (and later for other projects also). We think that periodical training sessions to keep other members educated should be good solution. Paying more attetion to sufficient documentation is also required. Having said that, both of these processes should follow KISS rule - training sessions and documentation should contain only necessary information and take as little time as possible.  
\item[Training] Team members are trained to perform training sessions (again, KISS highlighted).
\item[Deployment] Knowledge transfer sessions are incorporated into process flow.
\item[Trial] We check how does the knowledge transfer sessions influence the situation, if other team members feel educated enough to take care of possible emergency situation. If everything went according to the plan, measuring our metrics again we can see that total time is now 0 hours, we achieved our goal.
\end{enumerate}

In Post-Maturity phase it is important to analyze our strengths and weaknesses. As a strength we can notice that currently there is no possibility of emergency situation happening with no support available. On the other hand, we didn't perform knowledge sharing for different projects, so there is a risk of the same situation happening for different projects in the future (we should take that into account in further process improvement). Spending time on additional activity - sharing knowledge about projects can also be seen as a weakness, we can think how to optimize this process to take as little time as possible.
