%%%%%%%%%%%%%%%%%%%%%%%%%%%%%%%%%%%%%%%%%%%%%%%%%%%%%%%%%%%%%%%%%%%%%%%%%%%%%%%%
%2345678901234567890123456789012345678901234567890123456789012345678901234567890
%        1         2         3         4         5         6         7         8
% * <paulina@russak.biz> 2017-11-22T08:15:19.724Z:
%
% ^.
\documentclass[letterpaper, 10 pt, conference]{ieeeconf}  % Comment this line out
                                                          % if you need a4paper
%\documentclass[a4paper, 10pt, conference]{ieeeconf}      % Use this line for a4
                                                          % paper
\usepackage{paralist} 
\IEEEoverridecommandlockouts                              % This command is only
                                                          % needed if you want to
                                                          % use the \thanks command
\overrideIEEEmargins
% See the \addtolength command later in the file to balance the column lengths
% on the last page of the document



% The following packages can be found on http:\\www.ctan.org
%\usepackage{graphics} % for pdf, bitmapped graphics files
%\usepackage{epsfig} % for postscript graphics files
%\usepackage{mathptmx} % assumes new font selection scheme installed
%\usepackage{times} % assumes new font selection scheme installed
%\usepackage{amsmath} % assumes amsmath package installed
%\usepackage{amssymb}  % assumes amsmath package installed

\title{\LARGE \bf
Process Improvement, APIM and GQM+}

%\author{ \parbox{3 in}{\centering Huibert Kwakernaak*
%         \thanks{*Use the $\backslash$thanks command to put information here}\\
%         Faculty of Electrical Engineering, Mathematics and Computer Science\\
%         University of Twente\\
%         7500 AE Enschede, The Netherlands\\
%         {\tt\small h.kwakernaak@autsubmit.com}}
%         \hspace*{ 0.5 in}
%         \parbox{3 in}{ \centering Pradeep Misra**
%         \thanks{**The footnote marks may be inserted manually}\\
%        Department of Electrical Engineering \\
%         Wright State University\\
%         Dayton, OH 45435, USA\\
%         {\tt\small pmisra@cs.wright.edu}}
%}

\author{Szymon Planeta and Paulina Russak% <-this % stops a space
}


\begin{document}



\maketitle
\thispagestyle{empty}
\pagestyle{empty}


%%%%%%%%%%%%%%%%%%%%%%%%%%%%%%%%%%%%%%%%%%%%%%%%%%%%%%%%%%%%%%%%%%%%%%%%%%%%%%%%
\begin{abstract}

Process improvement is an important part of every project. This provide the higher quality of product or service and hence customer satisfaction. One of the valuable method of improvement processes are APIM and GQM+. In this paper we want to present the use cases of those methods based on our experienced in projects in software development.

\end{abstract}


%%%%%%%%%%%%%%%%%%%%%%%%%%%%%%%%%%%%%%%%%%%%%%%%%%%%%%%%%%%%%%%%%%%%%%%%%%%%%%%%
\section{Introduction}
Our main focus in this paper are two process improvement methods: APIM and GQM+. Process improvement is a continuous approach that provides elimination of weaknesses in the process. We need to improve processes to avoid unnecessary work or increase customer satisfaction. 

In section II we present our experience in software development projects. Section III contains three processes that we identified in the projects from previous part and issues in those processes that should could be resolved. Section IV is all about APIM model - here we describe how APIM model can be used in process 1 and process 3. In section V we present the GQM+ solution that could be used to improve process 1. Section VI provides information about similarities and differences in APIM and GQM+ methods. In section VII we describe a few tools that were presented in Jacob's book \cite{jacobs} and could be used in our process improvement cycles. At the end of this document there is short conclusion part where we contain the main thoughts about both methods.

\section{Experience}

Paulina was working for two and a half years as a c++ software developer in a big telecommunication company. She also was there for a one and a half year a Scrum Master and she take care about her team. She start working also on a big improvement project, but because of personal issue she has to stop working on that, but she'll come bat to it later. This project is described in detail below. She also work in a small family company and take care about setting up the web projects and take care about clients. One of the enjoyable enjoyable project that she was part of was project at the university where participants can run in safe environment agile project. 

Szymon was working for almost two years as a DevOps engineer in two big corporations. In both of these companies he worked according to Scrum methodology. Having said that, software development processes in those companies differed a lot. One of the processes will be described as an example in following chapters. Szymon was also a member of Paulina's team in the agile project at the university and that processes from this project will also be described as a common experience of both authors.

In this paper we want to focus mainly on agile projects such we can concentrate on APIM and GQM+ techniques.

\section{Processes description}

In this section we want to present the processes that we find in our project and what improvements should be done. Analyze of improvements contains improvements descriptions and predicted obstacles.

\subsection*{Process 1}

First process comes from project that one of us was participate in a big telecommunication company as a software developer. There was a few processes in that project, but we want to focus on process that handle the tasks flow in the system. This project was running in some kind of Scrum. At the beginning of the release Product Owner (PO) gets a list of requirements that should be done till specified date by ten teams. Each team has 6-8 members. Product Owner was setting each requirement in preliminary order and attributed tasks to teams. Each team take care of one domain. After this initial settings Product Owner gives a list of requires that can be handle in this release to Product Manger and release starts.

Project is running in two-weeks sprints. At the beginning of each sprint Product Owner gives a list of requirements or User Stories to the team. Team make a Scrum Poker and based on velocity and intuition select the proper amount of tasks to the sprint. When Product Owner accept the Sprint Backlog, sprint starts. As was mentioned above sprint takes two weeks and included activities such as: develop, test, integration. 

After sprint finish, team meets the Product Owner and shows him the work that is done and undone, velocity and time that team spends on maintenance. After this review section teams goes to the Retrospective meeting and Product Owner update his scheduled list and prepare new requirements to the next sprint.

Let's focus on Product Owner work and process from his perspective in that case. List of requirements is delivered in excel files and this is the main tool to schedule tasks. Each requirement has estimated time execution time (in hours) and the domain that should be done in. Teams during planning gives the PO their estimates (in story points) and then after each sprint they gives the PO number of effort that left.  Some of requirements should be done in particular order, and some of tasks appears in project late (because depends on external work). So there is a lot of issues that PO should concern about. In project there are two Product Owners, so they have to schedule work for five teams each in excel.

Based on assumption above there is a problem in planning part. First of all PO have to spend a lot of time to schedule tasks properly in excel. He don't see the whole picture, because of size of the project, so there are situations that team have anything to do in two sprints, because some other tasks are late and then they have to work overtime. 

Our proposition to improve is to give the POs new tools to schedule tasks. This tool should help with daily work and take care about iteration work. There is one big obstacle: people have a tendency to do exactly what is on the tools output. To avoid that PO should gain at least two different way to schedule work. It force him to think about the proposed solution, alternatively add his own improvements and choose the best option. 

\subsection*{Process 2}

Both of us was participating in a university project about Scrum. Project was about creating a web solution system to handling bank operations. We work in group of six and that contain people from different countries and cultures. 

The project flow starts from overall planning, during which the team describes the main project goal and set initial Project Backlog. After discussion with customer about vision of the product and priority, project starts. First iteration begins from Scrum Poker, where team estimate effort of particular User Stories, after this team decides which US will be in commitment and discusses about Sprint Backlog this with customer.

The next step is a developing part. During this part team starts coding tasks based on DoD. Members pick tasks from Sprint Board from column "TO DO" and then move them to the text columns: "DOING", "IN REVIEW", "DONE". Because of accommodation issues team members can't meet face to face, so there was remote work. Four times in sprint there was daily session by chat. During this session all  coworkers answer three question:
\begin{inparaenum}
\item "what did I do?",
\item "what will I do?",
\item "what obstacles do I have?"
\end{inparaenum} and discuss of necessary. \textit{może coś o pull requestach i code review? NOe wiem jak to ubrać w słowa, żeby nie było masło maslane}

After two weeks development part stops and starts revision and demo part. At the demo team shows work that was done to the customer, has discussion about customer needs and satisfaction, update Product Backlog and vision of the product. Then team goes to the Retrospective meeting and has discussion about good things and bad things that happened in the last sprint and what improvement should be done. In that moment first iteration stops. And second starts from planning again.

In this project there was three iterations. We don't finish all items from Product Backolg, but few more was added. We measure VSM, and it came out that there was some bottlenecks in the system. 

\textit{opis projektu, ustawienia początkowe, jak leciał scrum, zakończenie, probkemy: za mało usawień poczatkowych, brak analizy ryzyka, budzetu, etc,; w trakcie: za mało testów, problemy komunikacyjne}
APIM method is...

\subsection*{Project 1}
In the project 1, which description was presented above, one of the sources of the problem is planning part. Let's try using APIM to make some improvements to this process. First of all we should start from Launch phase. During this phase we have to convince the management that the improvement needs to be done. In this case we should collect some historical data and present them to the management and decision-makers. In our case data should contain information that planning is not going very well. One of the most valuable measurements in this scenario is how much time team spend on feature work. In this measurement we can clearly see, that in some sprints there are a little amount of time spend on feature work and in the next sprints there are even overtimes. 
This leads us to suppose that some tasks were late in system, so team have no job to do and then team has to handle the late and current tasks at the same time. Because this is a big project, we need to create an improvement process team separately to coordinate whole improvement process. During the Launch phase we also need to test if our improvement team works well. To do this we can go to the next phase where improvement team should do action plan - this will be their first task.

 %To sum up our improvement process is about making improvement in planning part in development cycle. We want to monitor and schedule tasks flow in agile project to equalize the team load. The problem occur because of Product Owners schedule tasks for whole release using excel and because of that they miss the whole picture. Work for teams is unevenly distributed over project life cycle. The new monitoring and schedule method is needed.

%\subsubsection*{Action Plan}
During Planning in Pre-Maturity phase we need to develop brief action plan. In this phase we have to focus on main goal and strategy, timeline with milestones, measures and risks. We should also describe Completion Criteria. The proposed action plan for process improvement is presented below: \\
\textbf{Title:}  Monitoring and scheduling task flow in agile project.\\
\textbf{Problem definition:} Product Owners schedule tasks for whole release using excel and because of that they miss the whole picture. Work for teams is unevenly distributed over project life cycle. The new monitoring and schedule method is needed.\\
\textbf{Goal:}  We want to monitor and schedule tasks flow in agile project to equalize the team load.\\
\textbf{Team:} Project improvement team is needed. It should contain about 4 to 5 people.\\
\textbf{People involved:} Development team and Product Owner.\\
\textbf{Strategy:} We need to take one development team to test the proposed solution. After each of three sprints of development, the schedule algorithm will change until it reaches completion criteria. After that new tool will be spreading to other teams.\\
\textbf{Desired Results:} The main expected result is equally distributed work during project life cycle. One of the key aspect in that scenario is also decrease in overtime working hours.\\
\textbf{Action List:} List of actions that should be performed to reach the goal. A sample action list is presented in Table \ref{tab:actionList}
\textbf{Risk:} Finding the right scheduling algorithm parameters in changing environment is one of the biggest risks that can increase budget and improvement time. The next aspect is that Product Owners can use the output of scheduling tool without second thoughts about that. When we are working in agile this is very undesirable strategy, because we want to follow the Manifesto rule: \textit{Individuals and interactions over processes and tools} \cite{manifesto}. We can decrease this risk by giving on the output two or more proposed schedule solution and let PO decide which one is the best based on intuition and people issue. \\
%\textbf{Timeline and milestones:} \textit{tu nie rozumiem dokładnie o jaki timeline chodzi}\\
\textbf{Completion Criteria:} Decrease overtime to x level. Equalize time spent on feature work to y level.

The next phase is Maturity phase. During this part we making improvements in cycle. Each cycle contains following steps: 
\begin{inparaenum}
\item awareness,
\item triange,
\item resolution,
\item training,
\item deployment,
\item trial.
\end{inparaenum}
Steps provided above are repeated until organization improvement goals are reached and \textit{an organization is ready for formal assessment} \cite{jacobs_short}. \textit{Awareness} phase is about setting small goals that should be reached after each iteration. In this phase we should also answer the question where are we now and what are our main weaknesses now \cite{jacobs} \cite{jacobs_short}. Then we have \textit{Triage} phase and during this phase we need to prioritize our improvement tasks and give them a value, to know which tasks should be done first to reach the goal quickly. The prioritization should be done based on business goals, project goals and process goals. Next step is \textit{Resolution} and this is about developing process. During this phase action plan is created for each task that involved people precisely know what is this improvement task about, why and how should be done. After the action plans are ready we need to train involved people how to use it. Fifth step is running the improvement in real life in the pilot group of involved people or other. There should be pilot mentor who take care about action plan and implementation. After \textit{Deployment} part is over we need to collect data and analyze current situation to the next improvements. \cite{jacobs} \cite{jacobs_short}.

Example of first iteration of Maturity phase in our project improvement process is presented below:

\begin{enumerate}
\item[Awareness] Based on our current situation we can say that there is some problem with planning part, but we don't have enough measurements to deeply describe the problem. 
\item[Triage] We have to move  tasks about collecting data to the top of the list, because we need to deeply understand the scale of the problem.
\item[Resolution] We need to describe the proposed measurements by making an action plan to each.
\item[Training] We need to train involved people how to use our proposed metrics based on action plan.
\item[Deployment] During this phase piloting team and Product Owner collect the measurement based on action plans. Mentor take care if everything goes according to plan.
\item[Trial] Collect data if measurement system works, if it needs any improvements and where are we now.
\end{enumerate}

Assuming that first Maturity cycle reached the Completion Criteria that measurement system works well and provide valuable information, we can run the next iteration. Based on historical data that we collected during project life cycle and our last iteration, we want to reach next step in our improvement process - give the Product Owner new tool to monitor and schedule tasks in system. Next cycle can looks like below:

\begin{enumerate}
\item[Awareness] Based on measurements we know that there are some sprints in which team doesn't spent time on features and on the other hand there are also sprints where people works overtime. The goal of this iteration is to provide feature work in each sprint on given level. We want to change the scheduling algorithm parameters and customize it.
\item[Triage] To reach the iteration goal, we need to move actions that are about setting the right parameters in new tool and implement it into system.
\item[Resolution] We need to set the initial parameters to the tool and write action plan how we want to use them. We should also prepare action plan from the Product Owner point of view.
\item[Training] We have to train the Product Owner how to use the new tool. 
\item[Deployment] Product Owner should use the new tool to schedule tasks for one team. Mentor should provide the highest quality information about action plan and instruction of usage. 
\item[Trial] Collect all data and think about how proper parameters are.
\end{enumerate}
In the next cycle we should focus on setting the proper parameters to the scheduling algorithm in proposed tool. Those iteration should be run for one pilot team until we reach the satisfactory results that meet Completion Criteria. After that goal is reached we can in the next cycle spread tool to another teams. 

At the end of APMI there is Post-Maturity phase. In this phase organization is ready for an formal audit. All cycle phases should be finished at this moment. During \textit{Asses} phase we need to analyze strength and weakens. In this case audit can be made by inside organization team. They need to compare measurements from the begging of the project the end of it. One of the most important thing that should be done is a survey measuring employee satisfaction.

The last step of Post-Maturity phase is to \textit{Improve}. Based on knowledge that we gain during this project we need to improve measurements, make changes in budget and update action plan. 



\begin{center}
\begin{table}
\caption{Example of action list in APIM for Project 1}
\label{tab:actionList}
\begin{tabular}{ | l | p{7cm} | }
 ID & Action \\ \hline
\rule{0pt}{3ex}1 & Measure and monitor overall task flow in the system \\
\rule{0pt}{3ex}2 & Implement new tool to Product Owner work\\
\rule{0pt}{3ex}3 & Set the proper parameters value in scheduling algorithm in tool\\ 
\end{tabular}

\end{table}
\end{center}



\subsection*{Project 3}

Maintenance is the process we look at in the third project. Applying APIM, we need to start with Launch phase again. To convince the management that the process needs an improvement, we need to present them some data. We can collect measurement, based on the working hours of team members, how much time certain projects are without responsible team members - when some failure happens there is no person responsible for the project that can take care of the accident. To give an example: Jim and Mary are the only workers responsible for project A. During 40 hours workweek either both or one of them is working all the time, excluding time window on Monday where from 12 a.m. to 4 p.m., when none of them is working. This leaves us with 4 hours time window every week, so that if an accident happens, there is nobody responsible to take care of it. We can think of it as 10\% sign chance of this situation happening (this is of course a simplification which takes into account this 4 hours time window and 40 hours workweek as the only factors). To further convice the management we can justify that having some services down for longer time can cause financial loss. We can give them real examples of situations that actually happened - like the need of calling the team members that were away on holidays to ask for help.

Brief action plan proposed for this process: \\
\textbf{Title:} Reducing total time of situations where services are live and no responsible team member is currently working.\\
\textbf{Problem definition:} Members of the team are working part-time and sometimes situations happen, that there is nobody to act in case of emergency situation.\\
\textbf{Goal:}  We want to reduce total time of situations like this happening.\\
\textbf{Team:} Project improvement team is needed. It should contain 2 people.\\
\textbf{People involved:} Development team.\\
\textbf{Strategy:} Knowledge about the projects should be spread across the team so that more members are aware of what to do in emergency situations.\\
\textbf{Desired Results:} Total time of service running without emergency help reduced to zero.\\
\textbf{Risk:} We can identify a risk that activities related to spreading knowledge about the projects across the team can take too much time (especially considering part-time workers) which can reduce overall productivity. Another risk is that hypothetically speaking there is a possibility that there still is no member responsible for the service (bad luck in schedule of workers so that nobody from the team is working at certain time, team members illness etc.).\\
%\textbf{Timeline and milestones:} \textit{tu nie rozumiem dokładnie o jaki timeline chodzi}\\
\textbf{Completion Criteria:} Total time of services without emergency help decreased to zero based on team members working hours.
 %To sum up our improvement process is about making improvement in planning part in development cycle. We want to monitor and schedule tasks flow in agile project to equalize the team load. The problem occur because of Product Owners schedule tasks for whole release using excel and because of that they miss the whole picture. Work for teams is unevenly distributed over project life cycle. The new monitoring and schedule method is needed.

Now comes Maturity phase. As described previously, we are following Maturity phase steps:
\begin{enumerate}
\item[Awareness] There is some hindrance in the maintenance process that sometimes causes too long downtimes of services in case of emergency situations. 
\item[Triage] We need to prioritize that will let us measure our current situation.
\item[Resolution] Preparation of action plan for measuring current situation. We need to measure occurences of the aforementioned situations based on team members working hours. This is simple task that can be automated, research needs to be done if proper software exists or if we need to implement it (which shouldn't be a tough task). 
\item[Training] Gathering information about working hours of team members is already in place, so team members do not need additional training. Process improvement team members need to understand the output of measurement software.
\item[Deployment] Deploy information gathering to our projects team.
\item[Trial] Insert collecting data from measurement system into process flow.
\end{enumerate}

After first iteration which was small and simple we have some meaningful data about current state of our processes, we can continue with analyzing it and persist in trying to reach the goal. Possible second iteration:
\begin{enumerate}
\item[Awareness] Analyzing collected data we can find out that each week there are in total 8 hours of situations that we want to avoid. We can notice that it is relevant only for projects X and Y.
\item[Triage] We prioritize knowledge sharing tasks for projects X and Y over other projects. Both projects are similarily important, but for project X total time is 6 hours so we set it to the highest priority.
\item[Resolution] We are preparing action plan for knowledge sharing for projects X and Y (and later for other projects also). We think that periodical training sessions to keep other members educated should be good solution. Paying more attetion to sufficient documentation is also required. Having said that, both of these processes should follow KISS rule - training sessions and documentation should contain only necessary information and take as little time as possible.  
\item[Training] Team members are trained to perform training sessions (again, KISS highlighted).
\item[Deployment] Knowledge transfer sessions are incorporated into process flow.
\item[Trial] We check how does the knowledge transfer sessions influence the situation, if other team members feel educated enough to take care of possible emergency situation. If everything went according to the plan, measuring our metrics again we can see that total time is now 0 hours, we achieved our goal.
\end{enumerate}

In Post-Maturity phase it is important to analyze our strengths and weaknesses. As a strength we can notice that currently there is no possibility of emergency situation happening with no support available. On the other hand, we didn't perform knowledge sharing for different projects, so there is a risk of the same situation happening for different projects in the future (we should take that into account in further process improvement). Spending time on additional activity - sharing knowledge about projects can also be seen as a weakness, we can think how to optimize this process to take as little time as possible.

\section{GQM+}

GQM+ Strategies approach was presented in  book \textit{Aligning Organizations through Measurement : the GQM+ Strategies Approach}. This technique is about reaching the organizational goals through measurements. The Goal in this case is \textit{what} the organization wants to achieve. Strategy is about \textit{how} we want to achieve our goals. Strategies are specific actions. Measurements are collected constantly to support making-decision process and to follow in the right direction \cite{basili}.

The GQM+ Strategies  Process consist of initial phase and six phases that are in cycle organized in three main sections. Let's begin from initializing phase. During initialization phase we need to specify the organization scope that will be improved. Then we need to prepare overall plan to our process, which consist analyze budget, risk, timeline with milestones. At the end of this we have to train involved people and take care about right minds settings.

Process cycle starts from develop section. During this section we create the GQM grid that contains goals, strategies and data to measure. To do this we need to 1) Characterize Environment by describe the current situation in organization and next there is phase for 2) Define Goals, Strategies and Measurements. In this moment we need to  
GQM+ for Project 1:

GQM Strategies Element:

\begin{itemize}
\item Organization Goal:
  \begin{itemize}
  \item[\textit{object: (analyze some)}] Planning in software development process \\
  We assumed that planning in our software development process need to be improved.
  \item[\textit{focus: (with respect to)}] effectiveness \\ We want to increase effectiveness of planning. 
  \item[\textit{magnitude:}] Reduce overtimes by X\% \\
  We reach the goal if overtimes will be reduced by X\%.
  \item[\textit{time frame:}] End of the release.\\
  We want to achieve the goal before end of current release.
  \item[\textit{organizational scope: (from the point of view of)}] Product Owner and development teams \\
  Main success of the achieving the goal depends on Product Owner and development teams.
  \item[\textit{constrains:}] Maintenance tasks \\
  Maintenance tasks occur in the system randomly and may influence to feature work.
  \item[\textit{relationships:}] Decrease defect level, cost reduction. \\
  We want to increase effectiveness of planning part, so we want to have at least stable level of maintenance work, so reaching decrease defect level help us. From another hand we want to achieve the goal by implement new tool, which can affect on budget and be in opposition to cost reduction goal.
  \end{itemize}
\item Context/Assumption: \\
In our organization we use the waterfall in a up level of planning and some variation about Scrum in development part. In this paper we want to focus on development part. List of requirements is known at the beginning of the project, but some requirements occur in the project at a specified time or have to be finish earlier, because of dependencies to other projects. Product Owner has to   create initial scheduling list of requirements and after each sprint he needs to update previous settings. Now he do all that work in excel, so he don't see the whole picture and because of that teams are uneven work loaded. 
\item Strategy (\textit{how?}):\\
The proposed solution is to implement new tool to the planning process. This tool should have to possibility to change scheduling algorithms parameters to customize output. Moreover tool has to provide minimum two version of output to avoid situation where Product Owner unthinkingly use the tool.
\end{itemize}

GQM Graph:
\begin{itemize}
\item[MG:] Equalize work load
	\begin{itemize}
	\item[Q:] How much people work overtime?
    	\begin{itemize}
    	\item[M:] counting overtime hours
    	\end{itemize}
    \item[Q:] How much time team spent on features?
    	\begin{itemize}
    	\item[M:] hours spent on features per sprint,
        \item[M:] overall feature time in project,
    	\end{itemize}
    \item[Q:] How team members feel about work load?
    	\begin{itemize}
    	\item[M:] satisfaction of the team
    	\end{itemize}
	\end{itemize}
\item[MG:] Increase effectiveness of Product Owner work.
	\begin{itemize}
	\item[Q:] How much time Product Owner spend on scheduling tasks?
    	\begin{itemize}
    	\item[M:] time spend on scheduling task at the beginning of the project,
        \item[M:] time spend on scheduling task at the beginning of each sprint,
    	\end{itemize}
    \item[Q:] How proper is plan created by Product Owner?
    	\begin{itemize}
    	\item[M:] number of late tasks,
        \item[M:] overtime hours,
        \item[M:] VSM: waiting time,
    	\end{itemize}
    \item[Q:] How people enjoyed their work?
    	\begin{itemize}
    	\item[M:] satisfaction of team,
        \item[M:] satisfaction of Product Owner,
        \item[M:] satisfaction of customer.
    	\end{itemize}
	\end{itemize}
\end{itemize}
\section{Compare GQM+ with APIM}

GQM+ is more about metrics and measurements

APIM is more about making improvements in small steps

Those methods are complementary and should be use together

\section{APIM Tools}

\begin{itemize}
\item APIM checklist

	In both processes described as examples we performed APIM. To keep track of actions performed we could make use of APIM checklist to have an overview on each APIM step and not forget about anything.

\item Audit Schedule, Plan, Checklist, Report

	In Post Maturity - Improve phase of APIM we could perform audit and develop continuous process improvement plan. Tools like audit schedule, plan, checklist and report will enable us to do that correctly and easier our work. Performing audit will help us ensure that created processes are being used. Creating audit schedule allows us to gain overview of distribution of actions in time. In our example (process 1) audit plan would be created and used by internal team of expert. They should take into consideration if completion criteria were met - in this case if the overtime working hours were reduced and if plans/procedures defined during prematurity phase were realized. One important aspect is to verify if all of the defined "involved people" were actually involved - Product Owner and developers team. Audit report will let us show results of audit in clear and concise way. It should contain comments and conclusions coming from the audit. Audit checklist (or Gap Analysis) would be other useful tools to look at while creating the audit report. In our example audit report could contain information about overtime hours reduction or if Product Owners are actually using newly developed tools - sometime even after process improvement actions people tend to go back to old habits. Following standardized templates of these documents will allow to easily compare audit results with previous ones or possible next ones.

\item Initial Executive Kickoff Meeting Agenda and Minutes

	These two tools are great to attract and keep executive staff attention and focus. Presenting agenda before the meeting let people realize purpose of this meeting, they can be prepared for what's it going to be. Example agenda (for our process 3) could contain bullets like: "vision: eliminate issues with emergency situations" or "objective: reduce to zero possibility of accident happening when no support is available". Meeting Minutes should be created to ensure that everyone that participated in the meeting is on the same page, understand everything the same way, to prevent misunderstanding. Minutes from this meeting should contain different items like attendees, discussions or action items. Example action items for process 3: "measure current time when projects lack support" or "spread knowledge about projects across team". Both of these documents should be kept as short and as simple as possible - the shortest the documents the more probable someone actually reads them.

\end{itemize}

\section{Conclusion}

In this paper we present the usage of APIM ang GQM+ methods in project that we were part of. Both methodologies based on continuous improvement philosophy and both are run in cycles. APIM method is more about making improvements in small steps what was taken from agile philosophy.  GQM+ from the other hand focus on measurements and deeply understanding the purpose of improvement. 


\begin{thebibliography}{99}

\bibitem{jacobs} Jacobs D., Accelerating Process Improvement Using Agile Techniques,  Auerbach Publications , 2005

\bibitem{manifesto} Beck Kent, Robert C. Martin and others, \emph{Agile Manifesto}, 2001, http://agilemanofesto.org

\bibitem{jacobs_short} Jacobs D., Accelerating Process Improvement Using Agile Techniques,  Cross Talk, 2004

\bibitem{basili} Basili V. [and six others]. Aligning Organizations through Measurement : the GQM+ Strategies Approach. Cham :Springer, 2014. Print.




\end{thebibliography}




\end{document}
