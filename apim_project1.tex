\subsection*{Project 1}
In project first, that description was presented above, one of source of the problem is planning part. Let's try using the APIM make some improvements to this process.  First of all there need to do the Launch phase. During this phase we have to convince the management that the improvement needs to be done. In this case we should collect some historical data and present them to the management and decision-makers. Data should contain information that planning is not going very well. One of the most valuable measurements in this case is how much time team spend on feature work. In this measurement we can clearly see that in some sprints there is a little amount of time spend on feature work and in the next sprints there are event overtimes. It can mean that some tasks was late in system and team have no job to do and then team has to handle the late tasks and current to do. Because this is a big project we need to create an improvement process team. \\

\subsubsection*{Action Plan}
There is shown the action plan\\
\textbf{Title:}  Monitoring and schedule task flow in agile project.\\
\textbf{Problem definition:} Product Owners schedule tasks for whole release using excel and because of that they miss the whole picture. Work for teams is unevenly distributed over project life cycle. The new monitoring and schedule method is needed.\\
\textbf{Team:} Project improvement teams is needed. It should contain about 4 to 5 people.\\
\textbf{Piloting Strategy:} We need to take one development team to test the proposed solution. After each three sprints of development, the schedule algorithm will change until it's reach completion criteria. After that new tool will be spreading to next teams.\\
\textbf{Desired Results:} The main expected result is equally distributed work during project life cycle. One of the key aspect in that scenario is also decrease overtime.\\
\textbf{Issues and Risk:} Finding the right scheduling algorithm parameters in changing environment is one of the biggest risk that can increase budget and improvement time. The next aspect is that Product Owners can use the  \\
\textbf{Timeline:} \\
\textbf{Deliverables:} 